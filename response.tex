\documentclass{article}

\usepackage{response}

% Additional packages used in this particular sample document.
\usepackage{fixltx2e}
\usepackage[T1]{fontenc}
\usepackage[utf8]{inputenc}
\usepackage[inline]{enumitem}
\usepackage{geometry}

\title{The \textsf{response} Package: Sample Usage}
\author{Sebastián González}

\begin{document}

\maketitle

\noindent
\textsl{(Note that the document must be compiled twice for the counters
  to be updated.)}

\section*{Response}

We would like to thank our anonymous reviewers for their efforts in
revising our first submission of the paper. We have considered all
remarks carefully, and addressed them to the best of our
possibilities.

The input from the reviews has been split into \arabic{totalremarks}
different remarks. Each remark is followed by our response and an
explicit status indication:
\begin{itemize}[nosep,label=--]
\newcommand{\total}[1]
  {\makebox[3ex][r]{\arabic{total#1}} remarks are flagged as
    \csname status#1\endcsname}
\item \total{solved}
\item \total{future}
\item \total{incorrect}\footnote{The remark seems to have overlooked
  a point made in the paper. We explain where and how the point is
  made.}
\item \total{disagree}\footnote{We ask the reviewer to reconsider
  the remark in the light of arguments and clarifications we give.}
\item \total{unsolved}\footnote{We could not accommodate the remark
  in the text due to blocking constraints (most likely lack of
  space, or else lack of time).}
\item \total{unspecific}\footnote{The remark is difficult to address
  without further input or feedback from the reviewer.}
\end{itemize}

\ifnumless{\value{totalresponses}}{\value{totalremarks}}
  {Watch out: there are a total of \arabic{totalremarks}
   remarks, but only \arabic{totalresponses} responses.}{}

\subsection*{Review 1}

\begin{remark}
  This paper positions itself very clearly from the point of [...]
\end{remark}
\begin{solved}
  We appreciate this positive feedback.
\end{solved}

\begin{remark}
  The section with case studies is not the strongest IMO [...]
\end{remark}
\begin{future}
  We agree that more validation is needed [...]
\end{future}

\begin{remark}
  The reader may easily get lost about [...]
\end{remark}
\begin{unspecific}
  All key ingredients are explicitly mentioned [...]
\end{unspecific}

\begin{remark}
  The general policy presented in section 2.5 looks OK but very naive.
  It will work for some kinds of applications but not for any.
\end{remark}
\begin{incorrect}
  The paper does not claim that [...]
\end{incorrect}

\begin{remark}
  The jQuery mobile example has a similar problem [...]
\end{remark}
\begin{disagree}
  That is not the point of this validation case [...]
\end{disagree}

\begin{remark}
  The introduction is rather weak [...]
\end{remark}
\begin{unsolved}
  We agree that we could motivate the problem further [...]  Much to
  our regret, we can barely fit the paper as it is now in 12 pages.
\end{unsolved}

\begin{remark}
  Finally, the paper is generally well-written, in an accessible
  style, although there is definite room for improvement here [...]
\end{remark}
\begin{unspecific}
  All key ingredients that make up the overall computation model are
  explicitly mentioned [...]
\end{unspecific}

\end{document}
